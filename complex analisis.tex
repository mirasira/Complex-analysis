\documentclass{article}

\usepackage{amssymb}
\usepackage{amsmath}
\usepackage{datetime}

\usepackage{pgfplots}
\pgfplotsset{compat=1.15}
\usepackage{mathrsfs}
\usetikzlibrary{arrows}

\usepackage{enumitem}

%definice čato používanejch příkazů
\newcommand{\mi}{\mathrm{i}}
\newcommand{\re}{\mathrm{Re}}
\newcommand{\im}{\mathrm{Im}}
\newcommand{\inv}{\mathrm{inv}}
\newcommand{\res}{\mathrm{res}}
%definice data formátu používanýho na úvodní stránce
\newdateformat{normaldate}{%
\twodigit{\THEDAY}.\twodigit{\THEMONTH}.\THEYEAR}

%klikací obsah
\usepackage{hyperref}
\hypersetup{
    colorlinks,
    citecolor=black,
    filecolor=black,
    linkcolor=black,
    urlcolor=black
}
\renewcommand*\contentsname{Obsah}
\date{\normaldate\today}
\title{Skupinový anal}
\author{Miroslav D.}
\begin{document}
\maketitle{}



\definecolor{qqqqff}{rgb}{0,0,1}
\definecolor{ffqqqq}{rgb}{1,0,0}
\begin{tikzpicture}[line cap=round,line join=round,>=triangle 45,x=1cm,y=1cm]
    \clip(-4.559768796141332,-5.51268778710339) rectangle (6.592170053800237,5.199306578866264);
    \draw [line width=2pt,color=ffqqqq] (0.8571428571428571,0.2857142857142857) circle (0.3194382824999699cm);
    \draw [line width=2pt,color=ffqqqq] (0.8461538461538461,0.3076923076923077) circle (0.34401045807689073cm);
    \draw [line width=2pt,color=ffqqqq] (0.8333333333333334,0.3333333333333333) circle (0.37267799624996495cm);
    \draw [line width=2pt,color=ffqqqq] (0.8181818181818182,0.36363636363636365) circle (0.4065578140908709cm);
    \draw [line width=2pt,color=ffqqqq] (0.8,0.4) circle (0.447213595499958cm);
    \draw [line width=2pt,color=ffqqqq] (0.7777777777777778,0.4444444444444444) circle (0.49690399499995325cm);
    \draw [line width=2pt,color=ffqqqq] (0.75,0.5) circle (0.5590169943749475cm);
    \draw [line width=2pt,color=ffqqqq] (0.7142857142857143,0.5714285714285714) circle (0.6388765649999399cm);
    \draw [line width=2pt,color=ffqqqq] (0.6666666666666666,0.6666666666666666) circle (0.7453559924999299cm);
    \draw [line width=2pt,color=ffqqqq] (0.6,0.8) circle (0.894427190999916cm);
    \draw [line width=2pt,color=ffqqqq] (0.5,1) circle (1.118033988749895cm);
    \draw [line width=2pt,color=ffqqqq] (0.3333333333333333,1.3333333333333333) circle (1.4907119849998598cm);
    \draw [line width=2pt,color=ffqqqq] (0,2) circle (2.23606797749979cm);
    \draw [line width=2pt,color=ffqqqq] (-1,4) circle (4.47213595499958cm);
    \draw [line width=2pt,color=ffqqqq] (3,-4) circle (4.47213595499958cm);
    \draw [line width=2pt,color=ffqqqq] (2,-2) circle (2.23606797749979cm);
    \draw [line width=2pt,color=ffqqqq] (1.6666666666666667,-1.3333333333333333) circle (1.4907119849998598cm);
    \draw [line width=2pt,color=ffqqqq] (1.5,-1) circle (1.118033988749895cm);
    \draw [line width=2pt,color=ffqqqq] (1.4,-0.8) circle (0.894427190999916cm);
    \draw [line width=2pt,color=ffqqqq] (1.3333333333333333,-0.6666666666666666) circle (0.7453559924999299cm);
    \draw [line width=2pt,color=qqqqff] (0.6,-0.2) circle (0.44721359549995787cm);
    \draw [line width=2pt,color=qqqqff] (0.5555555555555556,-0.2222222222222222) circle (0.49690399499995325cm);
    \draw [line width=2pt,color=qqqqff] (0.5,-0.25) circle (0.5590169943749475cm);
    \draw [line width=2pt,color=qqqqff] (0.42857142857142855,-0.2857142857142857) circle (0.6388765649999398cm);
    \draw [line width=2pt,color=qqqqff] (0.3333333333333333,-0.3333333333333333) circle (0.74535599249993cm);
    \draw [line width=2pt,color=qqqqff] (0.2,-0.4) circle (0.8944271909999162cm);
    \draw [line width=2pt,color=qqqqff] (0,-0.5) circle (1.118033988749895cm);
    \draw [line width=2pt,color=qqqqff] (-0.3333333333333333,-0.6666666666666666) circle (1.4907119849998598cm);
    \draw [line width=2pt,color=qqqqff] (-1,-1) circle (2.23606797749979cm);
    \draw [line width=2pt,color=qqqqff] (-3,-2) circle (4.47213595499958cm);
    \draw [line width=2pt,color=qqqqff] (5,2) circle (4.47213595499958cm);
    \draw [line width=2pt,color=qqqqff] (3,1) circle (2.23606797749979cm);
    \draw [line width=2pt,color=qqqqff] (2.3333333333333335,0.6666666666666666) circle (1.4907119849998598cm);
    \draw [line width=2pt,color=qqqqff] (2,0.5) circle (1.118033988749895cm);
    \draw [line width=2pt,color=qqqqff] (1.8,0.4) circle (0.8944271909999161cm);
    \draw [line width=2pt,color=qqqqff] (1.6666666666666667,0.3333333333333333) circle (0.7453559924999298cm);
    \draw [line width=2pt,color=qqqqff] (1.5714285714285714,0.2857142857142857) circle (0.63887656499994cm);
    \draw [line width=2pt,color=qqqqff] (1.5,0.25) circle (0.5590169943749476cm);
    \draw [line width=2pt,color=qqqqff] (1.4444444444444444,0.2222222222222222) circle (0.4969039949999532cm);
    \draw [line width=2pt,color=qqqqff] (1.4,0.2) circle (0.447213595499958cm);
\end{tikzpicture}
\newpage
\tableofcontents
    \section{týden}
        \subsection{Určení argumentu a absolutní hodnoty komplexního čísla}
            Absolutní hodnota
                \[|z| = \sqrt{(\mbox{Re}\:z)^2 + (\mbox{Im}\:z)^2}\]
                \[z = -3+3i,\quad |z|=\sqrt{(-3)^2+(3)^2}=3\]
            Argument
                \[\mbox{arg }z = \arctan\left(\frac{\mbox{Re }z}{\mbox{Im }z}\right)\]
                \[\mbox{arg}\left(\frac{\sqrt{3}+\mi}{1+\mi}\right)=
                \frac{\arg(\sqrt{3}+\mi)}{\arg(1+\mi)}=\arg(\sqrt{3}+\mi)-\arg(1+\mi)
                =\frac{\pi}{6}-\frac{\pi}{4}=-\frac{\pi}{12}\]
        \subsection{Jednoduché důkazy}
            Dokažte, že pro každé $z\in\mathbb{C}$ platí $|z|^2=z\overline{z}$
                \[|a+b|^2=(a+b\mi)(a-b\mi)\]
                \[\sqrt{a^2+b^2}^2=a^2+b^2\]
        \subsection{Moivreova věta}
                \[(\cos(x)+\mi\sin(x))^n=\cos(nx)+\mi\sin(nx)\]
                {\bf Příklad:}\\
                Nalezněte všechny hodnoty $\sqrt{3+4\mi}$
                \[\sqrt{3+4\mi}=\sqrt{5e^{\arctan\frac{4}{3}}}=(5(\cos(\arctan\left(\frac{4}{3}\right)+2k\pi)+
                \mi\sin(\arctan\left(\frac{4}{3}\right)+2k\pi)))^{\frac{1}{2}}\]
                \[=\sqrt{5}(\cos\left(\frac{\arctan\left(\frac{4}{3}\right)}{2}+k\pi\right)+
                \mi\sin\left(\frac{\arctan\left(\frac{4}{3}\right)}{2}+k\pi\right)),
                \qquad k = 0,1\]
        \subsection{Geometrické znázornění komplexních množin}
            Poloroviny
                \[\re z<2, \im z\geq 5\]
                \[|z+\mi|>|z-3-2\mi|\]
            Kružnice, mezikruží
                \[|z+2|<4\]
                \[2<|z+2|<4\]
            Úhly
            \[\frac{\pi}{2}<\arg(z)\leq\frac{3\pi}{4}\]
        \subsection{Konvergence řad}
            \subsubsection*{Nutná podmínka konvergence}
                \[\sum_{n=0}^\infty C_n \:\mbox{ konverguje} \Rightarrow \lim_{n\to\infty}C_n=0\]
            \subsubsection*{Podílové kritérium}
                \[\lim_{n\to\infty}\frac{|C_n+1|}{|C_n|}<1 
                \Rightarrow \sum_{n=0}^\infty |C_n| \mbox{konverguje}\]
                \[\lim_{n\to\infty}\frac{|C_n+1|}{|C_n|}>1 
                \Rightarrow \sum_{n=0}^\infty |C_n| \mbox{diverguje}\]
            \subsubsection*{Odmocninové kritérium}
                \[\lim_{n\to\infty}\sqrt{|C_n|}<1 
                \Rightarrow \sum_{n=0}^\infty |C_n| \mbox{konverguje}\]
                \[\lim_{n\to\infty}\sqrt{|C_n|}>1 
                \Rightarrow \sum_{n=0}^\infty |C_n| \mbox{diverguje}\]
                {\bf Příklad:}
                \[\sum_{n=0}^\infty(2+i)^{-n}\Rightarrow\lim_{n\to\infty}\sqrt[n]{|2+i|^{-n}}=
                |2+i|^{-1}=\frac{1}{\sqrt{5}}<1 \Rightarrow \mbox{Konverguje}\]
            \subsubsection*{Apolliniova kružnice}
            \[\lambda=\frac{|z-z_1|}{|z-z_2|}\]
    \section{týden}
        \subsection{Mocninné řady}
            \[\sum_{n=0}^\infty a_n(z-z_0)^n\]
            $z$ je proměnná\\
            $z_0$ je střed\\
            $a_n\in\mathbb{C}$ je n-tý koeficient
            \subsubsection*{Určení koeficientu příslušící každé k-té mocnině}
            {\bf Příklad:}
                \[\sum_{n=5}^{\infty}n^2(z-1+i)^{2n+1}=5^2(z-1+i)^{11}+
                6^2(z-1+i)^{13}+7^2(z-1+i)^{15}+\dots\]
                \[a_0=a_1=\dots=a_{10}=0\]
                \[a_k=0,\quad\forall k\in\mathbb{N}_0:\:k=2n,\quad n\in\mathbb{N}_0 \]
                \[a_k=\left(\frac{k-1}{2}\right)^2,\quad\forall k\in\mathbb{N}_0:
                \:k=2n+1,\quad n\in\mathbb{N}_0 \]
                \[a_{2n}=0\qquad\forall n\in\mathbb{N}_0\]
                \[a_{2n+1}=\begin{cases}
                    n^2\quad \forall n \in \mathbb{N}, \quad n\geq 5\\
                    0,\quad \mbox{pokud } n\in{0,1,2,3,4}
                \end{cases}\]

            \subsubsection*{Určení poloměru konvergence}
            \[\sum_{n=0}^\infty a_n(z-z_0)^n\quad \mbox{Konverguje absolutně pokud }|z-z_0|<R\]
            \[\sum_{n=0}^\infty a_n(z-z_0)^n\quad \mbox{Diverguje pokud }|z-z_0|>R\]
            \\
            {\bf Příklad:}
            \[\sum_{n=1}^{\infty}2^n(z-\mi)^n\:\Rightarrow\:\lim_{n\to\infty}\sqrt[n]{2^n|z-\mi|^n}
            =2|z-1|\]
            Řada konverguje absolutně pro $z\in\mathbb{C}:|z-\mi|<\frac{1}{2}$\\
            Řada diverhuje pro $z\in\mathbb{C}:|z-\mi|>\frac{1}{2}$\\
            Poloměr konvergence $R=\frac{1}{2}$
            \subsubsection*{Součet řad na kruhu konvergence}
                Záměna derivace a sumy
                    \[\frac{d}{dz}\left(\sum_{n=0}^\infty a_n(z-z_0)^n\right)=
                    \sum_{n=1}^\infty na_n(z-z_0)^{n-1}\]
                Záměna integrace a sumy
                    \[\int \left(\sum_{n=0}^{\infty}a_n(z-z_0)^n\right)dz=
                    \sum_{n=0}^{\infty}\frac{a_n}{n+1}(z-z_0)^{n+1}+c\]
                Základní součet geometrické řady
                    \[\sum_{n=0}^{\infty}z^n=\frac{1}{1-z}\]
                \\
                {\bf Příklad:}
                \[\sum_{n=0}^{\infty}nz^{n-1}=
                \frac{d}{dz}\left(\sum_{n=0}^{\infty}z^{n}\right)=
                \frac{d}{dz}\frac{1}{1-z}=\frac{1}{(1-z)^2}\]


    \section{týden}
        \subsection{Součet řad na kruhu konvergence 2}
            Obecný taylor
                \[f(z)=\sum_{n=0}^{\infty}\frac{f^{(n)}(z_0)}{n!}(z-z_0)^n\]
            Důležité funkce
                \[e^z=\sum_{n=0}^{\infty}\frac{z^n}{n!},\:z\in\mathbb{C},\qquad
                \ln(1-z)=-\sum_{n=1}^{\infty}\frac{z^n}{n},\:|z|<1\]
                \[\sin z =\sum_{n=0}^{\infty}\frac{(-1)^nz^{2n+1}}{(2n+1)!},\:z\in\mathbb{C},\qquad
                \sinh z =\sum_{n=0}^{\infty}\frac{z^{2n+1}}{(2n+1)!},\:z\in\mathbb{C}\]
                \[\cos z =\sum_{n=0}^{\infty}\frac{(-1)^nz^{2n}}{(2n)!},\:z\in\mathbb{C},\qquad
                \cosh z =\sum_{n=0}^{\infty}\frac{z^{2n}}{(2n)!},\:z\in\mathbb{C}\]
                \\
                {\bf Příklad:}
                \[\sum_{n=0}^{\infty}\frac{z^{2n}}{n!}=
                \sum_{n=0}^{\infty}\frac{z^{2^{n}}}{n!}=e^{z^2}\]
        \subsection{Rozvinutí funkcí na mocninou řadu}
            {\bf Příklad:}
            \[\frac{1}{9+z^2}=\frac{1}{9}\frac{1}{1+\frac{z^2}{9}}=
            \frac{1}{9}\frac{1}{1+\left(\frac{z}{3}\right)^2}=
            \frac{1}{9}\frac{1}{1-\left(\frac{zi}{3}\right)^2}=
            \frac{1}{9}\sum_{n=0}^{\infty}\left(\frac{zi}{3}\right)^{2n}=\]
            \[\frac{1}{9}\sum_{n=0}^{\infty}(-1)^n\left(\frac{z^2}{9}\right)^{n}\]
    \section{týden}
        \subsection{M\"{o}biova transformace}
        \[f(z):=\frac{az+b}{cz+d}\]
        Kruhová inverze
        \[\inv_k:\mathbb{C}\cup\{\infty\}\to\mathbb{C}\cup\{\infty\}\]
        \[\inv_k(z)=s+\frac{R^2}{\overline{z-s}},\:z\in\mathbb{C}\cup\{\infty\}\]

        Inverze vůči přímce\\
        V písemce budou jenom přímky rovnoběžné s Imaginární nebo Reálnou osou,
        ty jdou jednoduše spočítat přes osovou souměrnost, když tomu tak není tak 
        musíme použít tyhle vzorečky.
        \[x_p=\frac{b^2x_1-a(by_1+c)}{a^2+b^2},\qquad 
        y_p=\frac{a^2y_1-b(ax_1+c)}{a^2+b^2}\]
        \[\inv(x+\mi y)=2x_p-x_1+\mi(2y_p-y_1)\]
    {\bf Příklad:}
        Uvažujeme M\"{o}biovu transformace
        \[f(z):=\frac{z+\mi}{z-\mi}\]

        Na jaké množině v $\mathbb{C}\cup\{\infty\}$ zobrazí $f$ následující množina $M$?
        \[(a)M=\{z\in\mathbb{C}||z|\leq 1\}\]
        \[\mi\in k\Rightarrow\infty\in f(k)\: \mbox{ je přímka}\]
        Provedem inverzi 
        \[0+\frac{1^2}{\overline{\mi-0}}=\mi\]
        \[f(\mi)=0\]
        Je to přimka procházející 0, rovnoběžná na s imaginární osou
        Zobrazením bodu z kružnice zjistíme, o jakou se jedná polorovinu.
        \[f(0)=-1, \:\mbox{Jedná se o levou polorovinu}\]
        \[\{z\in\mathbb{C}|\re\: z \leq 0\}\]
    \section{týden}
        \subsection{Křivkový integrál}
        Dá se spočítat třema způsobama
        \begin{enumerate}
            \item Z definice, +funguje vždy -hodně práce
            \item Newton-Leibnitz +jednoduché spočítat -nefunguje vždy
            \item Cauchyho vzorec, reziduová věta - bude později
        \end{enumerate}

        \subsubsection{Z definice}
            \[\int_{C}f(z)dz=\int_{a}^{b}f(\varphi(t))\varphi'(t)dt=
            \int_C(u,-v)d\vec{S}+\mi\int_C(v,u)d\vec{S}\]
        %Jak parametrizovat?
        \\
        {\bf Příklad:}
        \[\int_C3z^2-2z\:dz\]
        kde křivka $C$ má parametrizaci $\varphi(t)=t+\mi t^2, t\in[0,1]$
        \[\int_0^1(3(t+\mi t^2)-2(t-\mi t^2))(2t\mi+1)=
        \int_0^1(3(t^2+it^3+it^3-t^4)-2t-2\mi t^2)(1+2\mi t)dt=\]
        \[=\int_0^1(-2t+3t^2-\mi t^2+6\mi t^3-3t^4)(1+2\mi t)dt=\]
        \[=\int_0^1(-2t+3t^2-2\mi t^2+6\mi t^3-3t^4-4\mi t^2+6\mi t^3+4t^3-12t^4-6it^5)dt\]
        \[=\int_0^1(-2t+3t^2-6\mi t^2+12\mi t^3+4t^3-15t^4-6\mi t^5)dt=\]
        \[=\left[-t^2+t^3-2\mi t^3+3\mi t^4+t^4-3t^5-\mi t^6\right]_0^1=
        (-1+1-2\mi+3\mi+1-3-\mi)=-2\]
        \subsubsection{Newton-Leibnitz}
            \[\int_Cf(z)dz=\underbrace{F(z_2)}_{\mbox{konec}}
            -\underbrace{F(z_1)}_{\mbox{začátek}}\qquad\mbox{ (F je primitivní funkce k $f$)}\]
        {\bf Příklad:}
            \[\int_C\frac{z+1}{z}dz\]
            Kde křivka C je úsečka [0, 2-i]. Počáteční bod C je 0 a koncový 2-i
            \\
            Najdeme primitivní funkci k $f$
            \[\int\frac{z+1}{z}\:dz=\int 1+\frac{1}{z}\:dz=z+\ln z\]
            Dosadíme do Newton-Leibnitze
            \[\mi+\frac{\pi}{2}+\mi+\frac{\pi}{2}=2\mi +\pi\]
    \section{týden}%6
        \subsection{Křivkový integrál 2}
        \subsubsection{Cauchyho vzorec}
            \[\frac{1}{2\pi\mi}\int_C\frac{f(z)}{z-z_0}dz=f(z_0)\]
        zobecněný Cauchyho vzorec
        \[\frac{n!}{2\pi\mi}\int_C\frac{f(z)}{(z-z_0)^{n+1}}dz=f^{(n)}(z_0)\]
        \\
        {\bf Příklad:}
        \[\label{eq1}\int_C\frac{1}{z(z^2-1)dz}\]
        $C$ je kladně orientovaná kružnice o poloměru $\frac{1}{2}$ a středu $1$.
        Singularity jsou 0,1 a -1, v kružnici leží jenom bod 1.
        \[\frac{f(z)}{z-1}=\frac{1}{z(z^2-1)}\:\Rightarrow\:f(z)=\frac{1}{z(z+1)}\]
        \[\int_c\frac{f(z)}{z-1}=2\pi\mi f(0)=\pi\mi\]

        {\bf Příklad:}
        \[\int_C\frac{z}{z^4-1}dz\]
        Kde křivka C má parametrizaci $\varphi(t)=a+ae^{\mi t}, t\in[0,2\pi], a>\frac{1}{2}$
        \\
        Singularity jsou v bodech $1,-1,\mi,-\mi$.
        Parametrizavoná křivka je půlkružnice se středem na reálné ose a koncovým
        bodem v počátku. Nejjedoduší bude použít $a=1\Rightarrow |z-1|=1$.
        Uvnitř se nachází pouze singularita $1$.
        \[\frac{f(z)}{z-1}=\frac{1}{(z^4)-1}\Rightarrow f(z)=\frac{1}{(z^2+1)(z+1)}\]
        \[\int_C\frac{\frac{1}{(z^2)(z+1)}}{z-1}=2\pi\mi f(1)=\frac{\pi\mi}{2}\]
    \section{týden}%7
        \subsection{Laurentovy řady}
        \[\sum_{n=-\infty}^{\infty}a_n(z-z_0)^n=
        \underbrace{\sum_{-1}^{n=-\infty}a_n(z-z_0)^n}_{\mbox{Hlavní část}}+
        \underbrace{\sum_{n=0}^{\infty}a_n(z-z_0)^n}_{\mbox{Regulární část}}\]
        \\
        {\bf Příklad:}
        Určit oblast konvergence Laurentovy řady
        \[\sum_{n=-\infty}^{-1}3^nz^{2n}+\sum_{n=0}^{\infty}\frac{n}{4^n}z^n\]
        \[\lim_{n\to\infty}\sqrt[n]{\left|3^nz^{2n}\right|}=3|z^2|\:\Rightarrow\:
        |z|<\frac{1}{\sqrt{3}}\]
        \[\lim_{n\to\infty}\sqrt[n]{\left|\frac{n}{4^n}z^n\right|}=
        \left|\frac{z}{4}\right|\:\Rightarrow\:|z|<4\]
        \[P\left(0;\frac{1}{\sqrt{3}};4\right)\]
        \\
        

    \section{týden}%9
        \subsection{Singularity}
            \begin{center}            
                \begin{tabular}{|cc|}
                    \hline
                    Typ&$\lim_{z\to z_0}f(z)$\\
                    \hline
                    odstranitelná singularita& $\in \mathbb{C}$\\
                    pól&$\infty$\\
                    podstatná singularita &neexistuje\\
                    \hline
                \end{tabular}
            \end{center}
            
            {\bf Příklad:}
                Klasifikujte všechny izolované singularity funkce
                \[f(z)=\frac{1}{\sin z}\]
                singularity: $\{k\pi,k\in z\},\:\infty$ není izolovaná singularita


        \subsection{Reziduum}
        \begin{enumerate}[label=\alph*.]
            \item v $ z_0\in\mathbb{C}$
            \item v $ \infty$
        \end{enumerate}
        Výpočet rezidua
        \begin{enumerate}
            \item Nalezení celé LŘ (zbytečně složité)
            \item Tvrzení (6.14, 6.17)
            \begin{enumerate}[label=\alph*.]
                \item $z_0\in\mathbb{C}$, pól f násobnosti $k\in\mathbb{N}$:
                    \[\res_{z_0}f=\frac{1}{(k-1)!}\lim_{z\to z_0}\frac{d^{k-1}}{dz^{k-1}}
                    \left[(z-z_0)^kf(z)\right]\]
                \item $\infty$ odstranitelná singularita:
                    \[\res_{\infty}f=\lim_{z\to\infty}z(f(\infty)-f(z))=\lim_{z\to\infty}z^2f'(z)\]
                \item $\infty$ pól násobnosti $k\in\mathbb{N}$:
                    \[\res f = \frac{(-1)^k}{(k+1)!}\lim_{z\to\infty}
                    \left[z^{k+2}\frac{d^{k+1}}{dz^{k+1}}f(z)\right]\]
            \end{enumerate}
            \item Tvrzení (6.15)
                f,g holomorfní v $z_0\in\mathbb{C},\:z_0$ je jednoduchý kořen $g$, pak
                \[\res_{z_0}\frac{f}{g}=\frac{f(z_0)}{g'(z_0)}\]
            \item Trvzení (6.16)
                f holomorfní v $z_0\in\mathbb{C}, g$ má v $z_0 $ jednonásobný pól pak
                \[\res_{z_0}(fg)=f(z_0)\res_{z_0}g\]
    \end{enumerate}

    {\bf Příklad:}
        Nalezněte rezidua funkce
        \[f(z)=\frac{z^2}{(z^2+1)^2}\]
        Ve všech izolovaných singularitách včetně bodu $\infty$.
        \\
        Singularity jsou $\pm\mi,\:\infty, \:\pm\mi$ jsou dvojnásobné singularity
        \[\lim_{z\to\infty}\frac{z^2}{(z^2+1)^2}=0\:\Rightarrow\: \infty \mbox{ je odstranitelná singularita}\]

        \[\underset{\mi}{\res}f=\frac{1}{1!}\lim_{z\to\mi}\frac{d}{dz}
        \left[(z-\mi)^2\frac{z^2}{(z^2+1)^2}\right]=\lim_{z\to\mi}\frac{d}{dz}
        \left[\frac{z^2}{(z^2+1)^2}\right]=\]
        \[\lim_{z\to\mi}\frac{2z(z+\mi)^2-2(z+\mi)z^2}{(z+\mi)^4}=
        \frac{2\mi(\mi+\mi)^2-2(\mi+\mi)\mi^2}{(\mi+\mi)^4}=-\frac{\mi}{4}\]
        \\
        \[\underset{-\mi}{\res}f=\frac{1}{1!}\lim_{z\to\mi}\frac{d}{dz}
        \left[(z+\mi)^2\frac{z^2}{(z^2+1)^2}\right]=\lim_{z\to\mi}\frac{d}{dz}
        \left[\frac{z^2}{(z^2-1)^2}\right]=\]
        \[\lim_{z\to\mi}\frac{2z(z-\mi)^2-2(z-\mi)z^2}{(z-\mi)^4}=
        \frac{2(-\mi)(-\mi-\mi)^2-2(-\mi-\mi)(-\mi)^2}{(-\mi-\mi)^4}=\frac{\mi}{4}\]
        \\
        \[\underset{\infty}{\res}=\lim_{t\to\infty}z^2f'(z)=
        \lim_{z\to\infty}z^2\frac{2z(z^2+1)^2-2(z^2+1)zz^2}{(z^2+1)^4}=0\]
    \section{týden}%9
    \subsection{Reziduová věta}
        \[\int_Cf(z)dz=2\pi\mi\sum_{j=1}^{k}\underset{z_j}{\res f}\]
        \begin{enumerate}
            \item res
        \end{enumerate}
    \subsection{Integrování reálných funkcí pomocí reziduové věty}
        \begin{enumerate}
            \item Racionální funkce $\frac{P}{Q}$, $\deg Q \geq \deg P+2$, na reálné ose
                    póly nejvýše násobnosti 1
                \[\int_{-\infty}^{\infty}\frac{P(x)}{Q(x)}dx=2\pi\mi\sum_{\underset{\underset{\im w>0}{Q(w)=0,}}{w\in\mathbb{C},}}
                \underset{z=w}{\res}\frac{P(z)}{Q(z)}+\pi\mi\sum_{\underset{Q(w)=0}{w\in\mathbb{R}}}\underset{{z=w}}{\res}
                \frac{P(z)}{Q(z)}\] 
            \item Funkce $R(x)e^{ix},\:R$ racionální funkce s reálnými koeficienty,
                $\lim_{z\to\infty} R(z)=0$, na reálné ose nejvýše póly násobnosti 1
                \[\int_{-\infty}^{\infty}R(x)e^{ix}dx=2\pi\mi\sum_{\underset{\underset{\im w>0}{w\mbox{ pól } R}}{w\in\mathbb{C}}}
                \underset{z=w}{\res}R(z)e^{\mi z}+\pi\mi\sum_{\underset{w\in\mathbb{R}}{w\mbox{ pól }R}}R(z)e^{\mi z}\]
                \begin{enumerate}[label=\alph*.]
                    \item R jako výše
                    \[\int_{-\infty}^{\infty}R(x)\cos x\:dx=\re \int_{-\infty}^{\infty}R(x)e^{\mi x}dx\]
                    \[\int_{-\infty}^{\infty}R(x)\sin x\:dx=\im \int_{-\infty}^{\infty}R(x)e^{\mi x}dx\]
                \end{enumerate}
            \item Funkce $R \cos x, \sin x, \: R$ racionální funkce definovaná na jednotokové kružnici v $\mathbb{R}^2$
                \[\int_{0}^{2\pi}R(\cos x,\sin x)dx=\int_CR\left(\frac{z^2+1}{2z},\frac{z^2-1}{2\mi z}\right)\cdot\frac{1}{\mi z}dz,\]
                Kde $C$ je kladně orientovaná jednotková kružnice v $\mathbb{C}$
        \end{enumerate}

    {\bf Příklad}
    \\
    Vypočtěte
        \[\int_{-\infty}^{\infty}\frac{1}{x^4+5x^2+4}dx\]
        \[f(z)=\frac{1}{z^4+5z^2+4}dz=\frac{1}{(z+\mi)(z-\mi)(z+2\mi)(z-2\mi)}\]
        Singularity, které nás zajímají jsou $\mi$ a $2\mi$
        \[\underset{z\to\mi}{\res}f=\lim_{z\to\mi}\frac{(z-\mi)}{(z+\mi)(z-\mi)(z+2\mi)(z-2\mi)}=\frac{1}{(2\mi)(3\mi)(-\mi)}=-\frac{1}{6}\mi\]
        \[\underset{z\to2\mi}{\res}f=\lim_{z\to2\mi}\frac{(z-2\mi)}{(z+\mi)(z-\mi)(z+2\mi)(z-2\mi)}=\frac{1}{(3\mi)(\mi)(4\mi)}=\frac{1}{12}\mi\]
        \[\int_{-\infty}^{\infty}\frac{1}{x^4+5x^2+4}dx=2\pi\mi\sum \res f=\frac{\pi}{6}\]
    \section{týden}%10S
        \subsection{Fourierova transformace}
        \[\hat{f}(\omega)=\int_{-\infty}^{\infty} f(t)e^{-i\omega t}dt,\qquad \omega \in \mathbb{R} \quad \mbox{přímá FT}\]
        \[\check{f}(\omega)=\int_{-\infty}^{\infty} f(t)e^{-i\omega t}dt,\qquad \omega \in \mathbb{R} \quad \mbox{inverzní FT}\]
        Základní gramatika
        \begin{center}
            \begin{tabular}{|c|c|}
                \hline
                posun ve vzoru& $\mathcal{F}\{f(t-a)\}=e^{-\mi\omega a}\hat{f}(\omega),\quad a\in\mathbb{R})$ \\
                posun v obraze&  $\mathcal{F}\{e^{\mi at}f(t)\}=\hat{f}(\omega-a), \quad a\in \mathbb{R}$\\
                škálování& $\mathcal{F}\{f(at)\}=\frac{1}{|a|}\hat{f}\left(\frac{\omega}{a}\right),\quad a\in\mathbb{R}\setminus\{0\}$\\
                konjungace& $\mathcal{F}\{\overline{f(-t)}\}=\overline{f(\omega)}$ \\
                \hline
            \end{tabular}
        \end{center}
    \section{týden}%11S
        \subsection{Fourierova transformace}
    \section{týden}%12S
        \subsection{Laplaceova transformace} 
    \section{týden}%13S
        \subsection{Transformace Z}
    \section{týden}%14S
        \subsection{Transformace Z}


\end{document}